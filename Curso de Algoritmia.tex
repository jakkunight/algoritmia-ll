% Document class:
\documentclass[a4paper, 12pt]{article}

% Used packages.
% Spanish language support:
\usepackage[spanish]{babel}

% UTF-8 support:
\usepackage[utf8]{inputenc}

% Graph support:
\usepackage{graphicx}

% Tables support:
\usepackage{tabularx}
\usepackage{array}
\usepackage{longtable}

% Geometry support:
\usepackage[a4paper, margin=1.5cm]{geometry}

% Margin notes support:
\usepackage{marginnote}

% Rotation support:
\usepackage{rotating}

% Color support:
\usepackage[dvipsnames]{xcolor}
\usepackage{tcolorbox}

% Global variables:
\title{\textbf{Curso de Algoritmia de Bajo Nivel}}
\author{Jakku Night}
\date{\today}

% Document:
\begin{document}
    \pagenumbering{gobble}
    \maketitle
    \newpage\section*{\textbf{\underline{Introducción}}}
    \subsection*{\textbf{Motivación del Curso}}
    \paragraph{
        Hace tiempo que quiero crear un curso de algoritmia para aquellos que quieren aprender a programar. 
        ¿Mi motivación? Mis ganas de enseñar a programar. Nada me motiva más que enseñar aquello que hubiera querido que me enseñen. 
        Me gusta mucho lo que hago, así que quiero compartir lo que me gusta con otras personas. Creo que ese es el pilar fundamental de 
        la programación. Por supuesto, también lo hago porque veo que es realmente necesario.
    }
    \subsection*{\textbf{El Problema de la enseñanza en la Algoritmia y la Programación}}
    \paragraph{
        Como mencioné, veo la realización de este curso como algo realmente necesario. No sólo en el sentido de que hayan pocos cursos 
        del tema por Internet, sino que también en las instituciones educativas es un dolor de cabeza aprender a programar. Hace no mucho, 
        cuando estuve haciendo un taller sobre robótica, me sorprendió lo rápido que los alumnos pudieron armar los distintos modelos de robot, 
        pero lo mucho que tardaron en comprender la lógica de la programación de los mismos. Algo andaba mal. Y efectivamente lo confirmé tras 
        hacer la mítica pregunta: ``¿Pueden explicarme cómo harían para hacer ustedes lo que quieren que haga el robot?'' No pudieron responder. 
        Me quedé bastante tiempo pensando en la reacción del curso. No hay que ser demasiado inteligente para darse cuenta de que si uno 
        no tiene claro cómo llevar una acción a cabo, tampoco va a tener claro como ordenarle a una máquina que haga lo que se supone que quieren 
        que haga. Lo peor de todo es que se supone que ellos son gente que estudia los algoritmos y ``saben'' programar. Realmente no sé como 
        hacen para programar, pero estoy seguro de que lo que sea que hagan, va a ser un completo desastre.
    }
    \paragraph{
        El principal dilema de los almnos es que no cuentan con una base sólida para enfrentar los ejercicios y retos planteados en el aula, 
        que, como se verá más adelante, son muy necesarios para poder hallar la solución a los diversos problemas. Para solucionar esto, 
        mi propuesta es realizar una nivelación inicial explicando las bases y fundamentos de mi metodología, luego entrar en los conceptos 
        fundamentales y así ir progresando en el uso de los conocimientos y habilidades adquiridos para construir una base que permita al 
        estudiante seguir por su cuenta.
    }
    \paragraph{
        El segundo problema es la carencia de lecciones o contenido teórico. La teoría es la guía por excelencia en ausencia del maestro, 
        por lo que su aprendizaje y asimilación no son pocas en comparación con la aplicación práctica. Es más, sin teoría no existe la 
        práctica. Esto deriva en un mal desarrollo de la capacidad de abstracción, que es fundamental para cualquier programador que se 
        precie. La solución es incluir más contenido teórico relevante y significativo para los estudiantes, a la par de que se estimula 
        su curiosidad y se sacia su sed de conocimiento.
    }
    \paragraph{
        El tercer problema, no tan evidente debido al ``maquillaje'' que hacen los distintos integrantes del sistema educativo, es la falta 
        de prácticas y la abundancia de malas prácticas. Esto se debe a la casi ausencia de conocimientos teóricos asociados directamente 
        con la práctica de la Algoritmia y la Programación. También es consecuencia directa de la mala base de conocimientos con la que 
        cuentan los estudiantes, ya que, insisto en esto, la capacidad del alumno en la programación depende enormemente de la base con la 
        que cuenta. La propuesta para esta ocasión es aprender a realizar proyectos simples con algún lenguaje de programación, como 
        Python, JavaScript, C++, etc.
    }
    \subsection*{\textbf{¿A quiénes va dirigido este libro?}}
    \paragraph{
        A todos aquellos que deseen aprender a programar y no cuenten con un bagaje técnico adecuado. Se recomienda poseer cierto conocimiento 
        matemático y geométrico de base, como saber realizar operaciones aritméticas fundamentales, calcular el perímetro y el área de 
        figuras planas, saber un poco de finanzas (ganancias, pérdidas, creación de presupuestos, etc) entre otros que, si bien no son 
        indispensables, pueden hacer la lectura mucho más ligera.
    }
    \subsection*{\textbf{¿Alguna recomendación para el resto del curso?}}
    \paragraph{
        Pueden visitar mi página de GitHub si desean encuntrar más acerca de los temas tratados en este curso o incluso si desean encontrar 
        más acerca de mí. También pueden consultar la bibliografía utilizada más abajo para profundizar sus conocimientos en algún 
        área mencionada en este curso o relacionada al mismo.
    }
    \newpage\section*{\textbf{\underline{Generalidades del Curso}}}
    \subsection*{\textbf{Método deductivo}}

    \newpage\section*{\textbf{\underline{Conceptos claves}}}
    \newpage\section*{\textbf{\underline{Resolución de Problemas}}}
    \newpage\section*{\textbf{\underline{La Computadora}}}
    \newpage\section*{\textbf{\underline{Generalidades de la Algoritmia y Programación}}}
    \newpage\section*{\textbf{\underline{Programación Tradicional o Lineal}}}
    \newpage\section*{\textbf{\underline{Programación Estructurada}}}
    \newpage\section*{\textbf{\underline{Programación Modular}}}
    \newpage\section*{\textbf{\underline{Programación Orientada a Objetos}}}
    \newpage\section*{\textbf{\underline{Programación Funcional}}}
    \newpage\section*{\textbf{\underline{Introducción a las Estructuras de Datos}}}
    \newpage\section*{\textbf{\underline{Listas}}}
    \newpage\section*{\textbf{\underline{Matrices}}}
    \newpage\section*{\textbf{\underline{Árboles}}}
    \newpage\section*{\textbf{\underline{Tablas}}}
    \newpage\section*{\textbf{\underline{Tablas Hash y Diccionarios}}}
    \newpage\section*{\textbf{\underline{Grafos}}}
    \newpage\section*{\textbf{\underline{Ordenamiento de Estructuras de Datos}}}
    \newpage\section*{\textbf{\underline{Búsqueda de Datos en Estructuras de Datos}}}
    \newpage\section*{\textbf{\underline{Sistema de Archivos}}}
    \newpage\section*{\textbf{\underline{Bases de Datos}}}
    \newpage\section*{\textbf{\underline{Programación Asíncrona}}}
    \newpage\section*{\textbf{\underline{Concurrencia}}}
    \newpage\section*{\textbf{\underline{Patrones de Diseño}}}
    \newpage\section*{\textbf{\underline{Programación en Red}}}
    \newpage\section*{\textbf{\underline{Introducción a la Arquitectura de Software}}}
    \newpage\section*{\textbf{\underline{Programación orientada a la Robótica}}}
    \newpage\section*{\textbf{\underline{Apéndice}}}
    \newpage\section*{\textbf{\underline{Bibliografía}}}
\end{document}
